% !TEX program = xelatex
\documentclass[14pt,a4paper]{article}

\usepackage[utf8]{inputenc}
\usepackage[russian]{babel}
\usepackage{geometry}
\usepackage{graphicx}
\usepackage{amsmath}
\usepackage{amsfonts}
\usepackage{amssymb}
\usepackage{hyperref}
\usepackage{listings}
\usepackage{xcolor}

\geometry{
    a4paper,
    left=2.5cm,
    right=2.5cm,
    top=2cm,
    bottom=2cm
}

\title{Автоматическое сопоставление резюме и вакансий}
\author{Фамилия Имя Отчество}
\date{\today}

\begin{document}

\maketitle

\tableofcontents

\section{Введение}
В современном мире, где рынок труда становится все более динамичным и конкурентным, 
процесс поиска работы и подбора персонала требует значительных временных и человеческих ресурсов. 
Автоматизация процесса сопоставления резюме и вакансий представляет собой актуальную задачу, 
решение которой может значительно повысить эффективность как для соискателей, так и для работодателей.

\section{Постановка задачи}
Основной целью данной работы является разработка системы автоматического сопоставления резюме 
и вакансий, которая позволит:
\begin{itemize}
    \item Автоматически анализировать структурированные и неструктурированные данные резюме
    \item Выделять ключевые навыки и требования из описаний вакансий
    \item Оценивать степень соответствия между резюме и вакансиями
    \item Предоставлять рекомендации по улучшению резюме для повышения релевантности
\end{itemize}

\section{Обзор существующих решений}
В настоящее время существует множество систем автоматического сопоставления резюме и вакансий, 
однако большинство из них имеют существенные ограничения:
\begin{itemize}
    \item Низкая точность сопоставления из-за использования простых алгоритмов
    \item Отсутствие учета контекста и специфики различных профессиональных областей
    \item Ограниченная поддержка русского языка
    \item Сложность интеграции с существующими HR-системами
\end{itemize}

\section{Методология}
Для решения поставленной задачи предлагается использовать комбинацию следующих подходов:
\begin{itemize}
    \item Обработка естественного языка (NLP) для анализа текстов резюме и вакансий
    \item Машинное обучение для классификации и ранжирования соответствий
    \item Векторные представления текста для семантического анализа
    \item Экспертные системы для учета специфики различных профессиональных областей
\end{itemize}

\end{document}
